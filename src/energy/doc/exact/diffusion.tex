\documentclass{article}
\usepackage[margin=1in]{geometry}
\usepackage{amsmath}
\usepackage{hyperref}
\parindent=0in \parskip=0.5\baselineskip

\usepackage{fancybox}
\usepackage{xcolor}
\newcommand{\highlight}[1]{{\color{red!80!black} \fbox{$ \displaystyle #1 $} }}

\begin{document}

\title{Exact solution for a simple IBVP with Dirichlet and Neumann BC}
\maketitle

\section{Introduction}
\label{sec:introduction}

This exact solution can be used to test the implementation of the
enthalpy and temperature column solvers, both in the ice and in the
bedrock.

This is sufficient to test the bedrock thermal unit code, but only
goes half way towards checking the enthalpy code: in these notes I
consider a simple \emph{diffusion only} problem without advection or
reaction terms.

\section{The problem}
\label{sec:problem}

\begin{align}
  \label{eq:1}
  \text{PDE:} \qquad & u_{t} = \alpha^{2}\, u_{zz},\quad 0 < z < L,\\
  \label{eq:2}
  \text{BCs:} \qquad &\left\{
                       \begin{aligned}
                         u(0,t) &= U_{0},\\
                         u_{z}(L,t) &= Q_{L},
                       \end{aligned}\right.\\
  \label{eq:3}
  \text{IC:} \qquad & u(z,0) = \phi(z).
\end{align}

In the PISM context
\begin{equation}
  \label{eq:14}
  \alpha^{2} = \frac{k}{\rho c_{p}},
\end{equation}
where $k$ is the thermal conductivity, $\rho$ is the density, and
$c_{p}$ is the specific heat capacity of ice.

To transform this problem with \emph{non-homogeneous} boundary conditions
into one with homogeneous ones I define $v(z,t)$ by subtracting the
steady state solution from $u$:
\begin{equation}
  \label{eq:4}
  v(z,t) = u(z,t) - \left( U_{0} + Q_{L}\cdot z \right).
\end{equation}

It is easy to check that $v_{t} = u_{t}$, $v_{zz} = u_{zz}$, and $v$ satisfies

\begin{align}
\label{eq:15}
  \text{PDE:} \qquad & v_{t} = \alpha^{2}\, v_{zz},\quad 0 < z < L,\\
  \label{eq:5}
  \text{BCs:} \qquad &\left\{
                       \begin{aligned}
                         v(0,t) &= 0,\\
                         v_{z}(L,t) &= 0,
                       \end{aligned}\right.\\
\label{eq:16}
  \text{IC:} \qquad & v(z,0) = \phi(z) - [U_{0} + Q_{L}z].
\end{align}

This new problem (equations (\ref{eq:15})--(\ref{eq:16})) can be solved using separation of variables, as follows.

Assume that $v(z,t)$ can be written as
\begin{equation}
  \label{eq:6}
  v(z,t) = Z(z)\cdot T(t),
\end{equation}
then the PDE (\ref{eq:15}) becomes
\begin{equation}
  \label{eq:7}
  ZT' = \alpha^{2}Z''T,
\end{equation}
or
\begin{equation}
  \label{eq:8}
  \frac{T'}{\alpha^{2}T} = \frac{Z''}{Z} = -\lambda^{2},
\end{equation}
for some constant $\lambda$, and so
\begin{align}
  \label{eq:9}
  T' + \lambda^{2}\alpha^{2}T &= 0,\\
  \label{eq:10}
  Z'' + \lambda^{2}Z &= 0.
\end{align}

Solving equations (\ref{eq:9}) and (\ref{eq:10}) gives
\begin{align}
  T(t) &= C_{1}e^{-(\lambda\alpha)^{2}t},\\
  Z(z) &= C_{2}\sin(\lambda z) + C_{3}\cos(\lambda z),
\end{align}
or
\begin{equation}
  \label{eq:11}
  v(z,t) = e^{-(\lambda\alpha)^{2}t}\left( A\sin(\lambda z) + B\cos(\lambda z) \right)
\end{equation}
Combining (\ref{eq:11}) with (\ref{eq:5}) gives
\begin{align}
  B &= 0,\\
  \cos(\lambda\cdot L) &= 0,
\end{align}
and so fundamental solutions have the form
\begin{align}
  \label{eq:12}
  v_{n}(z,t) &= \highlight{a_{n}\,e^{-(\lambda_{n}\alpha)^{2}t}\sin\left( \lambda_{n} z \right)},\\
  \lambda_{n} &= \highlight{\frac{1}{L} \left( \frac{\pi}{2} + n\,\pi \right),\quad n \in \mathbf{Z}}.
\end{align}

\end{document}
